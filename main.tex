%% A simple template for a term report using the Hagenberg setup
%% based on the standard LaTeX 'report' class
%%% äöüÄÖÜß  <-- no German umlauts here? Use an UTF-8 compatible editor!

%%% Magic comments for setting the correct parameters in compatible IDEs
% !TeX encoding = utf8
% !TeX program = pdflatex 
% !TeX spellcheck = en_US
% !BIB program = biber

\RequirePackage[utf8]{inputenc} % Remove when using lualatex or xelatex!
\RequirePackage{hgbpdfa}        % Creates a PDF/A-2b compliant document

\documentclass[english,notitlepage,smartquotes]{hgbreport}
% Valid options in [..]: 
%    Main language: 'german' (default), 'english'
%    Turn on smart quote handling: 'smartquotes'
%    APA bibliography style: 'apa'
%    Do not create a separate title page: 'notitlepage'
%%%-----------------------------------------------------------------------------

\graphicspath{{images/}}  % Location of images and graphics
\bibliography{references} % Biblatex bibliography file (references.bib)

%%%-----------------------------------------------------------------------------
\begin{document}
%%%-----------------------------------------------------------------------------

\author{Tobias Kothbauer, Veronika Leitner}                    % Your name
\title{Guide Guru - Interactive Travel Guide Project Report}	                 % or "Project Report"
\date{\today}

%%%-----------------------------------------------------------------------------
\maketitle
%%%-----------------------------------------------------------------------------

\begin{abstract}\noindent
In the dynamic landscape of modern travel, Guide Guru emerges as a individual solution, poised to redefine the way users embark on their adventures. Rooted in the fusion of cutting-edge technology and user-centric design philosophy, this project endeavors to craft a travel guide application that avoids limitations of traditional planning methods. 

\bigskip
\noindent At its core, Guide Guru is driven by the vision of empowering travelers with a personalized experience, one that seamlessly aligns with their interests, preferences and hobbies. Leveraging the capabilities of advanced large language models, the application will function as an intuitive digital companion, which adapts to the desires of each user. By fostering a tight relationship between technology and user input, Guide Guru endeavors to simplify the travel planning process, offering a platform where every journey is curated to reflect the personality of the individual. 

\bigskip
\noindent
\end{abstract}

%%%-----------------------------------------------------------------------------
\tableofcontents
%%%-----------------------------------------------------------------------------

%%%-----------------------------------------------------------------------------
\chapter{Aims and Context}
%%%-----------------------------------------------------------------------------

The core objective of this project is to develop an intuitive and adaptable travel guide application that seamlessly integrates user input with language processing capabilities. Through the implementation of user surveys, Guide Guru should gather and evaluate individual interests, ensuring that every travel recommendation is finely tuned to match the user's specific desires and hobbies.

The envisioned application gives users a tool with a simple interface, allowing them to effortlessly select destinations and specify personal interests, thereby generating travel guides curated to their liking.

Moreover, Guide Guru will offer the practical functionality of exporting personalized guides in PDF format, enabling users to conveniently access their curated recommendations on various devices and platforms.

Upon completion, Guide Guru targets delivering a travel companion, that eases the planning process and allows a high levels of customization and efficiency. By placing the user at the center of the experience, Guide Guru aims to enhance the quality of travel adventures, empowering individuals to craft marvelous journeys that align, with their interests and preferences.

To ensure that users interests are depicted in Guide Guru, an questionnaire was designed. XX participants allowed us an insight in their travel behavior and show us their hobbies and interests as well as their thoughts on how a good travel guide should look in their opinion. Through that Guide Guru is tailored to the needs of a broad audience, thus needing only small bits of information from the user.

%%%-----------------------------------------------------------------------------
\chapter{Project Details}
%%%-----------------------------------------------------------------------------

Describe important project steps, \eg, the rationale of the chosen architecture
or technology stack, design decisions, algorithms used, interesting challenges
faced on the way, lessons learned \etc

\bigskip

First foremost, we designed an questionnaire according to our thoughts on which information is most important to implement a simple, yet effective prototype with an highly individual outcome.

This survey included questions about the users preferences, ratings of relevance of different travel aspects and free text questions to grasp a short feeling for which hobbies are relevant for the users and as well as which information they want the travel guide to include.

%document the results of the survey

Meanwhile we researched about a potential technology stack. Two components were defined previously, as we wanted to improve our web-development skills with react we used this framework as our frontend technology. As well we already decided that we want to use ChatGPT from OpenAI as our large language model (LLM) of choice. Thankfully our supervisor provided us with an API key. 

However, after some iterations we decided to run our backend development with python as we found out that python is the well documented way to include a large language model into an web application. To allow this inclusion also the XXXXX Flask was used for handling requests and the CORS policies.

As mentioned above, many of our survey participants thought that images are a main criteria for a good travel guide. Thus the countless numbers of cities and places of interest that the user can generate with the input, we needed to use a way to access online data of multiple places. For that we used a combination of the Google Places API and Google Photos API to allow us to download and render images from the recommendations of users. With that technology we allow the users a simple but yet effective insight into the selected places of the travel guide. 

To validate the users input for the destinations we included the geonames API which is curated by user and lists millions of places. It also allows filtering for highly populated places. With this API we could include places with more than 1000 citizens, so a high variety of travel destinations can be used to generate the guide.




%%%-----------------------------------------------------------------------------
\chapter{System Documentation}
%%%-----------------------------------------------------------------------------

Give a well-structured description of the architecture and the technical design
of your implementation with sufficient granularity to enable an external person
to continue working on the project.

%%%-----------------------------------------------------------------------------
\chapter{Summary}
%%%-----------------------------------------------------------------------------

Give a concise (and honest) summary of what has been accomplished and what not. 
Point out issues that may warrant further investigation.

%%%-----------------------------------------------------------------------------
\appendix                                                   % Switch to appendix
%%%-----------------------------------------------------------------------------

%%%-----------------------------------------------------------------------------
\chapter{Supplementary Materials}
%%%-----------------------------------------------------------------------------

The appendix is a good place to attach a user guide, screenshots, installation
instructions, etc. Add a separate chapter for each major item.

%%%-----------------------------------------------------------------------------
\MakeBibliography[nosplit]
%%%-----------------------------------------------------------------------------

%%%-----------------------------------------------------------------------------
\end{document}
%%%-----------------------------------------------------------------------------
